% 6-inductor-properties

\section{Abstract}

For convenience we have put together some equations we have collected and consider useful.

\section{...}

\begin{eqnarray}
\mathcal{F} &=& N I \\
\mathcal{F} &=& H l \\
\mathcal{F} &=& \mathcal{R} \Phi \\
B &=& \mu H \\
B &=& \frac{\mu \mathcal{F}}{l} \\
\Phi &=& A B \\
\Lambda &=& \Phi N \\
\Lambda &=& L I \\
\end{eqnarray}

\begin{eqnarray}
L &=& A_L N^2 \\
A_L &=& \frac{\mu A}{l} = \frac{1}{\mathcal{R}} \\
\end{eqnarray}

\begin{eqnarray}
R &=& A_R N^2 \\
A_R &=& 8 \frac{\rho}{\varrho} \frac{r}{d \, l} \\
\end{eqnarray}

\begin{eqnarray}
\tau_L &=& \frac{L}{R} = \frac{A_L N^2}{A_R N^2} = \frac{A_L}{A_R} \\
\end{eqnarray}

\begin{eqnarray}
E &=& \frac{L I^2}{2} \\
E &=& \frac{{\Phi}^2}{2 A_L} \\
E &=& \frac{\mathcal{R}{\Phi}^2}{2} \\
E &=& \frac{A_L N^2 I^2}{2} \\
E &=& \frac{A_L R I^2}{2 A_R} \\
E &=& \frac{A_L P}{2 A_R} \\
E &=& \frac{\tau_L P}{2} \\
\frac{E}{P} &=& \frac{\tau_L}{2} \\
\end{eqnarray}

\begin{eqnarray}
F &=& \frac{B^2 A}{2 \mu_0} \\
F &=& \frac{\mu^2}{2 \, \mu_0 \, g^2} \, N^2 \, I^2 \, A \\
A_F &=& \frac{\mu^2 A}{2 \, \mu_0 \, g^2} \\
F &=& A_F \, N^2 \, I^2 \\
F &=& A_F \, \frac{P}{A_R} \\
\frac{F}{P} &=& \frac{A_F}{A_R} \\
\end{eqnarray}

\begin{eqnarray}
F &=& \frac{B^2 A}{2 \mu_0} \\
F &=& \frac{{\Phi}^2}{2 \mu A} \\
\end{eqnarray}

\begin{eqnarray}
{\Phi}^2 &=& 2 \mu A F \\
{\Phi}^2 &=& 2 A_L E \\
2 \mu A F &=& 2 A_L E \\
\frac{F}{E} &=& \frac{A_L}{\mu A} \\
\frac{F}{E} &=& \frac{1}{l} \\
F l &=& E \\
\end{eqnarray}

\section{Conclusion}

[TODO]

\appendix

\section{In Plain English}

[TODO]

\section{På Ren Svenska}

[TODO]


