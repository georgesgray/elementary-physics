\documentclass[]{../common/elementary-physics}

\title{6. Inductor Properties}
\date{2015-11-25 v0.1}

\begin{document}

\maketitle

\tableofcontents

% 6-inductor-properties

\section{Abstract}

For convenience we have put together some equations we have collected and consider useful.

\section{TODO}

The top equations has to be these:

\begin{subequations}
\begin{align}
B &= \mu H \\
\Phi &= A B
\end{align}
\end{subequations}

where $B$ is magnetic flux density in $T$, $H$ is magnetic field intensity in $A/m$ and $\mu$ is the permeability of the core.
$\Phi$ is the magnetic flux in $Wb$ and $A$ is the area in $m^2$.

\begin{subequations}
\begin{align}
\Lambda &= \Phi N \\
\Lambda &= L I
\end{align}
\end{subequations}

where $\Lambda$ is magnetic flux linkage in $Wb$, $N$ it the number of turns, $L$ is inductance in $H$ and $I$ is current in $A$.

\begin{subequations}
\begin{align}
\epsilon &= - \frac{\mathrm{d}\Lambda}{\mathrm{d}t} \\
\epsilon &= - N \frac{\mathrm{d}\Phi}{\mathrm{d}t} \label{faraday} \\
V &= L \frac{\mathrm{d}I}{\mathrm{d}t}
\end{align}
\end{subequations}

where $\epsilon$ is EMF in $V$.
Equation \eqref{faraday} is Faraday's law.

\begin{subequations}
\begin{align}
\mathcal{F} &= N I \\
\mathcal{F} &= H l \\
\mathcal{F} &= \mathcal{R} \Phi \\
\end{align}
\end{subequations}

\begin{subequations}
\begin{align}
L &= A_L N^2 \\
A_L &= \frac{\mu A}{l} = \frac{1}{\mathcal{R}}
\end{align}
\end{subequations}

\begin{subequations}
\begin{align}
R &= A_R N^2 \\
A_R &= 8 \frac{\rho}{\varrho} \frac{r}{d \, l}
\end{align}
\end{subequations}

\begin{subequations}
\begin{align}
\tau_L &= \frac{L}{R} = \frac{A_L N^2}{A_R N^2} = \frac{A_L}{A_R}
\end{align}
\end{subequations}

\begin{subequations}
\begin{align}
E &= \frac{L I^2}{2} \\
E &= \frac{{\Phi}^2}{2 A_L} \\
E &= \frac{\mathcal{R}{\Phi}^2}{2} \\
E &= \frac{A_L N^2 I^2}{2} \\
E &= \frac{A_L R I^2}{2 A_R} \\
E &= \frac{A_L P}{2 A_R} \\
E &= \frac{\tau_L P}{2} \\
\frac{E}{P} &= \frac{\tau_L}{2}
\end{align}
\end{subequations}

\begin{subequations}
\begin{align}
F &= \frac{B^2 A}{2 \mu_0} \\
F &= \frac{\mu^2}{2 \, \mu_0 \, g^2} \, N^2 \, I^2 \, A \\
A_F &= \frac{\mu^2 A}{2 \, \mu_0 \, g^2} \\
F &= A_F \, N^2 \, I^2 \\
F &= A_F \, \frac{P}{A_R} \\
\frac{F}{P} &= \frac{A_F}{A_R}
\end{align}
\end{subequations}

\begin{subequations}
\begin{align}
F &= \frac{B^2 A}{2 \mu_0} \\
F &= \frac{{\Phi}^2}{2 \mu A}
\end{align}
\end{subequations}

\begin{subequations}
\begin{align}
{\Phi}^2 &= 2 \mu A F \\
{\Phi}^2 &= 2 A_L E \\
2 \mu A F &= 2 A_L E \\
\frac{F}{E} &= \frac{A_L}{\mu A} \\
\frac{F}{E} &= \frac{1}{l} \\
F l &= E
\end{align}
\end{subequations}

\section{Conclusion}

[TODO]

\appendix

\section{In Plain English}

[TODO]

\section{På Ren Svenska}

[TODO]

\section{This Paper}

This is one paper from a collection of papers, all free to be downloaded and shared. If you have ideas how to enhance any of the papers, if you want to contribute, don’t hesitate to contact us at \url{hob.nilre@gmail.com}.\\
\\
The papers can all be found at:
\begin{itemize}
\item \url{https://sites.google.com/site/nilrehob/home/elementary-physics}
\item \url{https://independent.academia.edu/HobNilre/Papers}
\item \url{https://groups.yahoo.com/neo/groups/EVGRAY/files/Hob/}
\item \url{http://overunity.com/15796/elementary-physics-revisited/}
\item \url{http://idipsum.se/home/elementary%20physics.html}
\item \url{https://github.com/boherlin/elementary-physics/tree/master/papers}
\end{itemize}

They are updated with new versions in an unpredictable manner, possibly not on all sites but at least on the last two sites in the list, make sure you always have the latest version!
Their \LaTeX source-codes can be found at \url{https://github.com/boherlin/elementary-physics/tree/master/src}.
All papers, but not all versions, have been stamped at \url{http://www.OriginStamp.org}.\\
\\
If you enjoyed this paper, found value in it or want to help us, please consider giving us a donation in bitcoin, this is our address:

\begin{figure}[ht] \centering
	\includegraphics[]{../common/1B79p75vQw4Rb1GQdmGYpDapFwEytFJDqw} \caption{1B79p75vQw4Rb1GQdmGYpDapFwEytFJDqw}
\end{figure}


\printbibliography

\end{document}

